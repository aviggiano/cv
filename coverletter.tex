\documentclass[11pt]{article}


\usepackage[utf8x]{inputenc}
\usepackage[T1]{fontenc}
\usepackage{lmodern}
\usepackage{marvosym}
\usepackage{ifpdf}
\ifpdf
  \usepackage[pdftex]{graphicx}
\else
  \usepackage[dvips]{graphicx}\fi

\pagestyle{empty}

\usepackage[margin=2cm]{geometry}
\setlength{\parindent}{0pt}
\addtolength{\parskip}{6pt}

\def\firstname{Antonio Guilherme}
\def\familyname{Ferreira Viggiano}
\def\FileAuthor{\firstname \familyname}
\def\FileTitle{Personal Statement \textemdash~\firstname \familyname}
\def\FileSubject{Cover letter}
\def\FileKeyWords{\firstname \familyname, Personal Statement}
%\def\today{November 9, 2012} %% remove if not applicable byAFGV

\renewcommand{\ttdefault}{pcr}
\newcommand*{\NEWLINE}{\vspace{0.75em}}

\usepackage{url}
\urlstyle{tt}
\ifpdf
  \usepackage[pdftex,pdfborder=0,breaklinks,baseurl=http://,pdfpagemode=None,pdfstartview=XYZ,pdfstartpage=1,colorlinks=false]{hyperref} %% added hidelinks to hyperref byAGFV
  \hypersetup{
    pdfauthor = \FileAuthor,%
    pdftitle = \FileTitle,%
    pdfsubject = \FileSubject,%
    pdfkeywords = \FileKeyWords,%
    pdfcreator = \LaTeX,%
    pdfproducer = \LaTeX}
\else
  \usepackage[dvips]{hyperref}
\fi
\usepackage{helvet}
\usepackage{ragged2e}

\begin{document}
\sffamily % for use with a résumé using sans serif fonts;

{\bfseries MSc in Mathematical and Theoretical Physics \textemdash~University of Oxford}
\NEWLINE{}\NEWLINE{}

%br-engineering
%fr-mathematics, more theoretical, some didnt like

I would like to undertake a Master of Science in Mathematical and Theoretical Physics, as I am very dedicated to pursue a research-related career in mathematics and its applications to physical systems. This is why I have chosen to go to the University of Oxford, not only for its well established academic prestige, but also for its unique postgraduate programme that combines the most up to date theoretical physics with its mathematical foundations. The opportunity to read among a wide variety of subjects, and to follow a specific pathway that suits my research interests is what excites me the most about this course. \NEWLINE{}

My interest in theoretical physics and applied mathematics started during my double degree program at the École Centrale Marseille, in France. Although I had already studied classical mechanics, thermodynamics and electromagnetism in Brazil, at the University of São Paulo, these courses were more oriented towards their practical applications to engineering. The common core subjects of the French curriculum, on the other hand, had a greater mathematical rigor and a more in-depth study of physical concepts. There I was able to study complex analysis, continuum mechanics and quantum mechanics, for example, which will be essential to the degree I am applying for. \NEWLINE{}

Most of these subjects were very challenging, not only because they were graded on oral presentations, differently from what I was familiar in Brazil, but also because I did not meet some of the prerequisites. Because of the ``Molecular Structures and Properties'' course, some of the Brazilians and I started a study group to catch up with the rest of the class. We reviewed the week's lessons, rehearsed for the oral assignment and discussed advanced exercises, which were kept on a shared folder together with other subjects. Initially only available to us, these activities later expanded to all exchange students and then to the rest of the school, and some of my lecture notes are still being used nowadays, over six years after its inception. In the end, I was able to get one of the best grades of the class while helping others pass the exam. \NEWLINE{}


Being in contact with more advanced physics topics sparked my curiosity to go further on what was being taught in class. It all started with the book ``QED: The Strange Theory of Light and Matter'' by Richard Feynman, which opened my eyes to some of the counterintuitive behaviors of subatomic interactions. Nevertheless, the popular science writing did not fully satisfy my appetite for a thorough understanding of the subject, leading me to the online textbook ``Quantum Mechanics for Engineers'', from the Florida State University, for a more formal and self-contained explanation of quantum physics. Later, I started studying the ``Quantum Computation and Quantum Information Theory'' course from the Carnegie Mellon University, since I wanted to have a better comprehension of quantum computers. At the same time, I read the ``Astrophysics for People in a Hurry'' book from Neil deGrasse Tyson, a playful and educational story that explains in simple terms some of the concepts of astrophysics and cosmology. More recently, I have been taking the ``Quantum Mechanics'' online series of courses from the Massachusetts Institute of Technology, which is helping me solidify my knowledge in this area with different exercises and assignments.
\NEWLINE{}

\noindent\rule{\textwidth}{1pt}

Because of my double degree program, I had the opportunity to study a large selection of courses, ranging from mathematics and natural sciences to robotics, computer science and management. While some of them are more related to this degree than the others, I believe the common core subjects from both USP and ECM will be the most useful for the “Teorica Universalis” pathway. This includes many modules of general physics and classical mechanics taught in Brazil, covering solid and fluid mechanics and electromagnetism, as well as more advanced topics addressed in France, such as continuum mechanics, quantum mechanics and complex and numerical analysis. 

\noindent\rule{\textwidth}{1pt}

1/ Intention (I would like to study XYZ) + why I decided to apply to this
specific university.
2/ What sparked my interest (quantum) + education (ECM, USP) + how it relates - absorb ideas
3/ commitment to subject beyond degree
4/ experience and how it relates to application (independence/leadership revmob, endurance/innovation beluga) - sustain/reasoning
5/ Purpose for doing course (quantum computing = future) - intended pathway, specific areas of interest, specialization
5/ Career goals? What will be used from this course, how does it help? Future


1. Intro: Intention (I would like to study XYZ), why I decided to do a master’s (motivations for applying). What I have done until now (relevant experience and education), 
2. Motivation + understanding for the area of study
3. Commitment to the subject beyond requirements of the degree course
4. Capacity for sustained and intense work + reasoning ability
5. Ability to absorb abstract ideas and at a rapid pace
6. Intended pathway through the programme
7. Specific Areas of interest \& intention to specialise in
8. Conclusion
\NEWLINE{}


Why are you applying? / want to join this univ
How your studies/school link to what you are applying to
Wider reading -> outside curriculum related to degree
Other things / support -> trips, reading, museum, etc
Hobbies, interests \NEWLINE{}

What degree
What first authors attracted you to the subject?
Topics you like
Schools/subjects contribute
Career in mind \NEWLINE{}

You know what you want to study (love subject, passionate, what inspires you) 
Make everything relevant (I love academia, they don’t care about work experience, how xp makes you a good student, no general work xp), evaluate things
Intro - unique thing about yourself, personal, powerful, passion and relate to subject, better than any other candidate
Intro, book read, talk about subjects (A level), degree, work xp (1 par) linked to subject.
Specific examples (of how subjet/xp help or is important to degree), give quotes, chapters, names, authors, researches, … 
Reading (mention text read, actually read)
Conclusion - what to do in the future, what want to study specifically
Redraft and redraft and redraft, get many people to read it
Question every single sentence and paragraph -- what does it mean?
\NEWLINE{}

1. Why study that subject in that university, 2. Academic achievements, 3. Extracurricular achievements
Passionate about the subject
Highlight academic achievements
Extracurricular: Leadership, determination. Keep brief (admissions → academic)
\NEWLINE{}

Hardworking, ambitious / creative open-minded
Tell your personal story, moments where you really wanted to learn about it. Questions you want to answer, make statement personal to you
Communicate intellectual property, how teachable you are, when you don’t know the answer you are willing to ask/learn, hungry for knowledge
\NEWLINE{}

80\% academic, interests, achievements, relevant extracurricular, 20\% unrelated extracurricular
Discard information not needed
\NEWLINE{}

Evidence of achievements to support, richness, details, truthful, honest, knowledge of subject, promote yourself, 
NO: negativity, yourself/teacher, tutor, avoid franchise, not cliché, not too abstract quoting favorite lines of literature
Really positive doc, emphasis good applicant
\NEWLINE{}

Person really interested in course, what they have done in their lives, where they want to go in their studies
Impress passion for subject, care for personal statement, skills for being a good student
Be specific, tell about you, not general statement
Teamwork, commitment (long period of time), voluntary work (qualities), how they thought the subject would be
Start early! Particular interest in a subject, evidence that. Not just list experiences
\NEWLINE{}

Why do you want to study this course?
Why have you chosen this particular university?
What initially sparked your interest in the area? (where it first developed)
What experiences do you have? (first or second hand) - studies / articles
What is your overall purpose for doing this course? (quantum comp = future?)
What skills are required, and do you have them? (previous experiences) - eg resilience, patience, ...
What specific techniques or specialized skills do you have? - eg techniques
What you will get out of this course? (purpose, how this course will help you in the other areas you want to improve on)
What are your future career goals, and how will this course help?
\NEWLINE{}

State your intention right at the start (I would like to study X at Y because I’m interested in developing the Z skill)
Adopt the ‘hourglass’ structure (start generally, narrow in the middle with specifics, expand out again about interest/ in the field and)
Be sure to ‘talk up’ the institute you’re applying for (facilities, renowned for department) - no generic things
State what you have learned from your experiences
Don’t overuse ‘passion’, ‘enthusiastic’, or ‘interested’
Don’t sound like a ‘know it all’
Be as authentic and as genuine as you can (no cliché, overused)
\NEWLINE{}




\end{document}
