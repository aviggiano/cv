\documentclass[10pt]{article}

\usepackage[utf8x]{inputenc}
\usepackage[T1]{fontenc}
\usepackage{lmodern}
\usepackage{marvosym}
\usepackage{ifpdf}
\ifpdf
  \usepackage[pdftex]{graphicx}
\else
  \usepackage[dvips]{graphicx}\fi

\pagestyle{empty}

\usepackage[margin=2cm]{geometry}
\setlength{\parindent}{0pt}
\addtolength{\parskip}{6pt}

\def\firstname{Antonio Guilherme }
\def\familyname{Ferreira Viggiano}
\def\FileAuthor{\firstname \familyname}
\def\FileTitle{\firstname\familyname~\textemdash~Personal Statement}
\def\FileSubject{Cover letter}
\def\FileKeyWords{\firstname\familyname, Personal Statement}

\renewcommand{\ttdefault}{pcr}
\newcommand*{\NEWLINE}{\vspace{0.75em}}

\usepackage{url}
\urlstyle{tt}
\ifpdf
  \usepackage[pdftex,pdfborder=0,breaklinks,baseurl=http://,pdfpagemode=None,pdfstartview=XYZ,pdfstartpage=1,colorlinks=false]{hyperref} %% added hidelinks to hyperref byAGFV
  \hypersetup{
    pdfauthor = \FileAuthor,%
    pdftitle = \FileTitle,%
    pdfsubject = \FileSubject,%
    pdfkeywords = \FileKeyWords,%
    pdfcreator = \LaTeX,%
    pdfproducer = \LaTeX}
\else
  \usepackage[dvips]{hyperref}
\fi
\usepackage{helvet}
\usepackage{ragged2e}

\begin{document}
\sffamily % for use with a résumé using sans serif fonts;

{\bfseries \FileTitle}
\NEWLINE{}\NEWLINE{}

I would like to undertake a Master of Science in Mathematical and Theoretical Physics, as I am very determined to leave a legacy for future generations by participating in the development of science. Seeking to study in one of the leading universities in the world, I have chosen to go to the University of Oxford, not only for its well established academic prestige but also for its unique postgraduate programme. The opportunity to read the most up to date mathematical and theoretical physics, presented through a wide variety of subjects, together with the idea of following a specific pathway that suits my research interests is what excites me the most about this course. \NEWLINE{}

My desire to leave a mark in the world has always followed me throughout my academic and career choices. It all began when I decided to study Mechatronics Engineering at the University of São Paulo in Brazil. Rejecting to follow my family's tradition in medicine was a difficult decision, but at the time I believed I could impact more lives by inventing something truly different. Being accepted in one of the most reputable institutions in my country did not stop me from expanding my horizons, which is why I applied for a double degree program in General Engineering at the École Centrale Marseille, in France. The exchange program presented me to a different way of thinking, and my familiarity with theoretical physics did not emerge until I was introduced to the French curriculum. There I was able to study engineering topics through a greater mathematical rigor and a more in-depth study of hypothetical physical systems. Among many abstract lectures, I learned continuum mechanics, which introduced me to the beauty of tensor calculus' notations, as well as complex analysis, that proved useful to the study of quantum mechanics at the university. \NEWLINE{}

From the collision of these two complementary education systems, I started to develop a greater interest in theoretical subjects over practical ones. Among computer science and management books needed for my career as a software engineer, I preferred to study applied mathematics in my free time, always questioning myself if I should favor my passion over my job so as to achieve my goal of transforming society. It all started with Richard Feynman's ``QED: The Strange Theory of Light and Matter'', which have opened my eyes to many counterintuitive behaviors of subatomic interactions. While being a great introduction to the field, the popular science writing did not satisfy my appetite for a thorough understanding of small particles, as the explanations were too naive and near magical. This discomfort has led me to self-study most of the online textbook ``Quantum Mechanics for Engineers'', from the Florida State University, for an extensive explanation of quantum physics. Later, reflecting my desire to expand my knowledge about the cosmos, I have read the ``Astrophysics for People in a Hurry'' book from Neil deGrasse Tyson, an educational text that presents various astrophysics concepts in simple terms. More recently, I have been following the ``Quantum Mechanics'' series of online courses from the Massachusetts Institute of Technology, which is helping me to review some familiar topics through different lenses.
\NEWLINE{}

During my time as a programmer, ...
In order to succeed at this Oxford master course, I believe that, besides my motivation, independence and resilience are also required. This is where my work experience helps me the most. When I joined a technology startup as a software engineering intern, I had to understand the system's architecture without much support from the busy team. Because of my outstanding performance, I was invited by the chief executive officer to co-found a big data company in the development of an analytics database. In the early days of the business, when our only customer was threatening to leave the platform, we spent two months working more than twelve hours a day non-stop, including weekends and holidays, so we could deliver all requested features. This dedication was essential for maintaining the client and demonstrates that I can face any amount of stress to achieve my goals. \NEWLINE{}

Recently I have left my position as tech lead of the organization in order to pursue a career in science. My desire to understand the fundamental laws of the universe have outgrown my aptitude for programming, and after reviewing that all my free time activities were focused in expanding this passion, it was a straightforward decision to join the Mathematical and Theoretical Physics master's degree. Since I wish to get a broader comprehension of nature, I intend to choose the Generalist Theoretical Physicist pathway of the course. I am particularly curious about general relativity and quantum field theory, as I believe these are both groundbreaking theories that will follow up on my educational background and reading interest. After completing the programme, I plan to apply for a Ph.D. position at Oxford or another top-tier institute and continue to chase my goal of being a researcher in the field.

\end{document}
