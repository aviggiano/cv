\documentclass[10pt]{article}

\usepackage[utf8x]{inputenc}
\usepackage[T1]{fontenc}
\usepackage{lmodern}
\usepackage{marvosym}
\usepackage{ifpdf}
\ifpdf
  \usepackage[pdftex]{graphicx}
\else
  \usepackage[dvips]{graphicx}\fi

\pagestyle{empty}

\usepackage[margin=2cm]{geometry}
\setlength{\parindent}{0pt}
\setlength{\parskip}{1.75em}

\def\firstname{Antonio Guilherme }
\def\familyname{Ferreira Viggiano}
\def\subj{Personal Statement}
\def\FileAuthor{\firstname\familyname}
\def\FileTitle{\firstname\familyname~\textemdash~\subj}
\def\FileSubject{\subj}
\def\FileKeyWords{\firstname\familyname, \subj}

\renewcommand{\ttdefault}{pcr}

\usepackage{url}
\urlstyle{tt}
\ifpdf
  \usepackage[pdftex,pdfborder=0,breaklinks,baseurl=http://,pdfpagemode=None,pdfstartview=XYZ,pdfstartpage=1,colorlinks=false]{hyperref} %% added hidelinks to hyperref byAGFV
  \hypersetup{
    pdfauthor = \FileAuthor,%
    pdftitle = \FileTitle,%
    pdfsubject = \FileSubject,%
    pdfkeywords = \FileKeyWords,%
    pdfcreator = \LaTeX,%
    pdfproducer = \LaTeX}
\else
  \usepackage[dvips]{hyperref}
\fi
\usepackage{helvet}
\usepackage{ragged2e}

\begin{document}
\sffamily % for use with a résumé using sans serif fonts;

{\bfseries \FileTitle}

I would like to undertake a Master of Science in Mathematical and Theoretical Physics, as I believe this course will help me achieve my goal of leaving a legacy to the world through the development of science. Seeking to study in one of the leading universities in the planet, I have chosen to go to the University of Oxford, not only for its well established academic prestige but also for its unique postgraduate programme. The opportunity to read from the most up to date mathematical and theoretical physics, presented through a wide variety of subjects and pathways, is what excites me the most about this degree.

My desire to leave a mark in humanity did not appear until I started studying Mechatronics Engineering at the University of São Paulo in Brazil, hoping to create something truly different that could impact many people's lives. Wanting to expand my horizons and increase my chances of being a successful inventor, I applied for a double degree in General Engineering at the École Centrale Marseille, in France. There I was exposed to many abstract lectures containing more rigorous mathematical studies of physical systems, such as continuum mechanics, which made me familiar tensor calculus' beautiful notations, as well complex analysis and its applications to quantum mechanics, all of which will be fundamental to the degree I am applying to.

From the collision of these two complementary education systems, I started to develop a deeper interest in theoretical subjects than in practical ones, which made me look for additional sources of knowledge outside the classroom. It all began with Richard Feynman's ``QED: The Strange Theory of Light and Matter'', which have opened my eyes to many more counterintuitive behaviors of subatomic interactions. While being a great introduction to the field, the popular science writing did not satisfy my appetite for a thorough understanding of small particles, as the explanations were too naive and near magical. This discomfort has led me to self-study most of the online textbook ``Quantum Mechanics for Engineers'', from the Florida State University, for an extensive explanation of quantum physics. Later, desiring to expand my knowledge about the cosmos, I have read the ``Astrophysics for People in a Hurry'' book from Neil deGrasse Tyson, an educational text that presents various astrophysics concepts in simple terms. More recently, I have been following the ``Quantum Mechanics'' series of online courses from the Massachusetts Institute of Technology, which is helping me review some familiar topics through different lenses.

After coming back from the exchange program, I decided to work in a place where I could impact the largest number of people. As a result, I joined a technology startup as a software engineering intern, acting as one of its core members in the definition of its culture and programming standards. Because of my outstanding performance, I was invited by the chief executive officer to co-found a big data company. Beyond the implementation of traditional algorithms and data structures, we also reviewed many open source projects and research papers, which culminated in the development of an innovative architecture, as well as in our own data parser, serializer, and messaging protocol, among other original solutions. The great success and usage of the database rapidly incurred in a large number of customer demands. At the time, my co-founder and I spent over two months working more than twelve hours a day non-stop, including weekends and holidays, so that we could deliver all requested features. This dedication was essential for maintaining the clients and it demonstrates that I can face any amount of stress or work to fulfill my objectives.

Despite my accomplishments as an entrepreneur, having helped to establish a million-dollar company with more than a dozen employees, I have decided to leave my position as tech lead of the organization in order to pursue a career in science. Over time, I have concluded that even though it is possible to change the world in industry, the impact of one's actions are not as permanent as they are when applied to academic disciplines. Even the most successful companies in history vanish when compared to the immortality of the discoveries of Isaac Newton or Albert Einstein, who have not only provided us with a greater understanding of physics but who have also developed whole new fields of mathematics on the way to their findings. My desire to become part of this select group is what motivates me to join the Mathematical and Theoretical Physics master's degree. Since I wish to get a broader comprehension of nature, I intend to choose the Generalist Theoretical Physicist pathway of the course. I am particularly curious about general relativity and quantum field theory, as I believe these are both groundbreaking theories that still have much to be studied. After completing the programme, I intend to apply for a Ph.D. position at Oxford or another top-tier institute and continue to follow my ambition of influencing future generations as a researcher in mathematics and physics.

\end{document}
