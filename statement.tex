\documentclass[10pt]{article}

\usepackage[utf8x]{inputenc}
\usepackage[T1]{fontenc}
\usepackage{lmodern}
\usepackage{marvosym}
\usepackage{ifpdf}
\ifpdf
  \usepackage[pdftex]{graphicx}
\else
  \usepackage[dvips]{graphicx}\fi

\pagestyle{empty}

\usepackage[margin=2cm]{geometry}
\setlength{\parindent}{0pt}
\setlength{\parskip}{1.5em}

\def\firstname{Antonio Guilherme }
\def\familyname{Ferreira Viggiano}
\def\subj{Personal Statement}
\def\FileAuthor{\firstname\familyname}
\def\FileTitle{\firstname\familyname~\textemdash~\subj}
\def\FileSubject{\subj}
\def\FileKeyWords{\firstname\familyname, \subj}

\renewcommand{\ttdefault}{pcr}

\usepackage{url}
\urlstyle{tt}
\ifpdf
  \usepackage[pdftex,pdfborder=0,breaklinks,baseurl=http://,pdfpagemode=None,pdfstartview=XYZ,pdfstartpage=1,colorlinks=false]{hyperref} %% added hidelinks to hyperref byAGFV
  \hypersetup{
    pdfauthor = \FileAuthor,%
    pdftitle = \FileTitle,%
    pdfsubject = \FileSubject,%
    pdfkeywords = \FileKeyWords,%
    pdfcreator = \LaTeX,%
    pdfproducer = \LaTeX}
\else
  \usepackage[dvips]{hyperref}
\fi
\usepackage{helvet}
\usepackage{ragged2e}

\begin{document}
\sffamily % for use with a résumé using sans serif fonts;

{\bfseries \FileTitle}

Being an engineer by training and a programmer by profession, I now aspire to be a mathematician and physicist, seeking to leave a mark in the world through the development of science. My desire to change society goes way back to my undergraduate degree, when I applied to Mechatronics Engineering at the University of São Paulo, in Brazil, hoping to invent something that could impact many people's lives. During the course, I received a double degree in General Engineering at the École Centrale Marseille, in France, where I took many abstract lectures containing more rigorous mathematical studies of physical systems. From the collision of these two education systems, I started to develop a deeper interest in theoretical subjects over practical ones, which would follow me until this day.

Oxford's unique Mathematical and Theoretical Physics postgraduate programme is an ideal choice to expand my long-standing interest in natural philosophy and to prepare me for research in the field. Because of my general knowledge of quantum electrodynamics and Feynman diagrams, from Richard Feynman's ``QED: The Strange Theory of Light and Matter'', I am particularly interested in the courses on Quantum Field Theory. I believe my solid background in Quantum Mechanics, from the French curriculum and from my self-study of Florida State University's ``Quantum Mechanics for Engineers'' online textbook, will help me succeed in these courses, together with my previous knowledge of Complex Analysis' path integrals and other mathematical tools. I am also curious about General Relativity, as I think this will be an exciting follow up on Special Relativity and Continuum Mechanics that I have had in my exchange program. I expect my understanding of Tensor Calculus and Electromagnetism to be very useful to this subject, since it seems to combine all these disciplines in a coherent manner.                

From the collision of these two complementary education systems, I started to develop a deeper interest in theoretical subjects than in practical ones, which made me look for additional sources of knowledge outside the classroom. It all began with Richard Feynman's ``QED: The Strange Theory of Light and Matter'', which have opened my eyes to many more counterintuitive behaviors of subatomic interactions. While being a great introduction to the field, the popular science writing did not satisfy my appetite for a thorough understanding of small particles, as the explanations were too naive and near magical. This discomfort has led me to self-study most of the online textbook ``Quantum Mechanics for Engineers'', from the Florida State University, for an extensive explanation of quantum physics. Later, desiring to expand my knowledge about the cosmos, I have read the ``Astrophysics for People in a Hurry'' book from Neil deGrasse Tyson, an educational text that presents various astrophysics concepts in simple terms. More recently, I have been reading through the ``Quantum Computation and Quantum Information Theory'' lecture notes from the Carnegie Mellon University, which is helping me understand in more detail the particularities of quantum computers.

After coming back from the exchange program, I decided to work in a place where I could impact the largest number of people. As a result, I joined a technology startup as a software engineering intern, acting as one of its core members in the definition of its culture and programming standards. Because of my outstanding performance, I was invited by the chief executive officer to co-found a big data company. Beyond the implementation of traditional algorithms and data structures, we also reviewed many open source projects and research papers, which culminated in the development of an innovative architecture, as well as in our own data parser, serializer, and messaging protocol, among other original solutions. The great success and usage of the database rapidly incurred in a large number of customer demands. At the time, my co-founder and I spent over two months working more than twelve hours a day non-stop, including weekends and holidays, so that we could deliver all requested features. This dedication was essential for maintaining the clients and it demonstrates that I can face any amount of stress or work to fulfill my objectives.

Despite my accomplishments as an entrepreneur, having helped to establish a million-dollar company with more than a dozen employees, I have decided to leave my position as technical lead of the organization in order to pursue a career in science. Over time, I have concluded that even though it is possible to change the world in the private sector, the impact of one's actions are generally not as permanent as when they are applied to academic disciplines. Even the most successful companies in history vanish when compared to the immortality of the discoveries of Isaac Newton or Albert Einstein, who have not only provided us with a greater understanding of physics but who have also developed whole new fields of mathematics on the way to their findings. My desire to become part of this select group is what motivates me to join the Mathematical and Theoretical Physics master's degree. Since I wish to get a broader comprehension of nature, I intend to choose the Generalist Theoretical Physicist pathway of the curriculum. I am particularly curious about quantum field theory and general relativity, as I believe these are both groundbreaking theories that still have much to be studied. After completing the course, I intend to apply for a Ph.D. programme at Oxford or another top-tier institute and continue to follow my ambition of influencing future generations as a researcher in mathematics and physics.

\end{document}
