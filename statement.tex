\documentclass[10pt]{article}

\usepackage[utf8x]{inputenc}
\usepackage[T1]{fontenc}
\usepackage{lmodern}
\usepackage{marvosym}
\usepackage{ifpdf}
\ifpdf
  \usepackage[pdftex]{graphicx}
\else
  \usepackage[dvips]{graphicx}\fi

\pagestyle{empty}

\usepackage[margin=2cm]{geometry}
\setlength{\parindent}{0pt}
\addtolength{\parskip}{6pt}

\def\firstname{Antonio Guilherme }
\def\familyname{Ferreira Viggiano}
\def\FileAuthor{\firstname \familyname}
\def\FileTitle{\firstname\familyname~\textemdash~Personal Statement}
\def\FileSubject{Cover letter}
\def\FileKeyWords{\firstname\familyname, Personal Statement}

\renewcommand{\ttdefault}{pcr}
\newcommand*{\NEWLINE}{\vspace{0.75em}}

\usepackage{url}
\urlstyle{tt}
\ifpdf
  \usepackage[pdftex,pdfborder=0,breaklinks,baseurl=http://,pdfpagemode=None,pdfstartview=XYZ,pdfstartpage=1,colorlinks=false]{hyperref} %% added hidelinks to hyperref byAGFV
  \hypersetup{
    pdfauthor = \FileAuthor,%
    pdftitle = \FileTitle,%
    pdfsubject = \FileSubject,%
    pdfkeywords = \FileKeyWords,%
    pdfcreator = \LaTeX,%
    pdfproducer = \LaTeX}
\else
  \usepackage[dvips]{hyperref}
\fi
\usepackage{helvet}
\usepackage{ragged2e}

\begin{document}
\sffamily % for use with a résumé using sans serif fonts;

{\bfseries MSc in Mathematical and Theoretical Physics \textemdash~University of Oxford}
\NEWLINE{}\NEWLINE{}

I would like to undertake a Master of Science in Mathematical and Theoretical Physics, as I am very enthusiastic about pursuing a research-related career in this field. Seeking to study in one of the leading universities in the world, I have chosen to go to the University of Oxford, not only for its well established academic prestige but also for its unique postgraduate programme. The conjunction of the most up to date mathematical and theoretical physics, presented through a wide variety of subjects, and the opportunity to follow a specific pathway that suits my research interests is what excites me the most about this course. \NEWLINE{}

My familiarity with theoretical physics began during my double degree program at the École Centrale Marseille, in France. Although I had already studied applied mathematics and physics in Brazil, at the University of São Paulo, the courses contained an expressive amount of laboratory sessions, being more oriented towards their practical applications to engineering. The French curriculum, on the other hand, had a greater mathematical rigor and a more in-depth study of physical systems. There I was able to learn, among many abstract lectures, continuum mechanics, which introduced me to the beauty of tensor calculus' index notation, as well as complex analysis, that proved useful to the study of quantum mechanics, all of which will be fundamental to the degree I am applying for. \NEWLINE{}

Most of these subjects were very challenging, not only because they were graded on oral presentations, unlike in Brazil, but also because some of their prerequisites were not met by exchange students. One particular demanding course, ``Molecular Structures and Properties'', motivated some of the Brazilians and me to create a study group in order to catch up with the rest of the class. We reviewed the week's lessons, rehearsed for the oral assignments and discussed advanced exercises. These activities then expanded to all exchange students and then to the rest of the school, and some of my lecture notes are still being used nowadays, over six years later. In the end, I was able to get one of the best grades in the class while helping others pass the exam. \NEWLINE{}

Being in contact with more advanced physics topics sparked my curiosity to go further on what was being taught in class. It all started with Richard Feynman's ``QED: The Strange Theory of Light and Matter'', which have opened my eyes to many counterintuitive behaviors of subatomic interactions. While being a great introduction to the field, the popular science writing did not satisfy my appetite for a thorough understanding of small particles, as the explanations were too naive and near magical. This discomfort has led me to self-study most of the online textbook ``Quantum Mechanics for Engineers'', from the Florida State University, for an extensive explanation of quantum physics. Later, reflecting my desire to expand my knowledge about the cosmos, I have read the ``Astrophysics for People in a Hurry'' book from Neil deGrasse Tyson, an educational text that presents various astrophysics concepts in simple terms. More recently, I have been following the ``Quantum Mechanics'' series of online courses from the Massachusetts Institute of Technology, which is covering some familiar topics in a different manner.
\NEWLINE{}

In order to succeed at this Oxford master course, I believe that, besides my motivation, independence and resilience are also required. This is where my work experience helps me the most. When I joined a technology startup as a software engineering intern, I had to understand the system's architecture without much support from the busy team. Because of my outstanding performance, I was invited by the chief executive officer to co-found a big data company in the development of an analytics database. In the early days of the business, when our only customer was threatening to leave the platform, we spent two months working more than twelve hours a day non-stop, including weekends and holidays, so we could deliver all requested features. This dedication was essential for maintaining the client and demonstrates that I can face any amount of stress to achieve my goals. \NEWLINE{}

Recently I have left my position as tech lead of the organization in order to pursue a career in science. My desire to understand the fundamental laws of the universe have outgrown my aptitude for programming, and after reviewing that all my free time activities were focused in expanding this passion, it was a straightforward decision to join the Mathematical and Theoretical Physics master's degree. Since I wish to get a broader comprehension of nature, I intend to choose the Generalist Theoretical Physicist pathway of the course. I am particularly curious about general relativity and quantum field theory, as I believe these are both groundbreaking theories that will follow up on my educational background and reading interest. After completing the programme, I plan to apply for a Ph.D. position at Oxford or another top-tier institute and continue to chase my goal of being a researcher in the field.

\end{document}
