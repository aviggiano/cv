\documentclass[10pt]{article}

\usepackage[utf8x]{inputenc}
\usepackage[T1]{fontenc}
\usepackage{lmodern}
\usepackage{marvosym}
\usepackage{ifpdf}
\ifpdf
  \usepackage[pdftex]{graphicx}
\else
  \usepackage[dvips]{graphicx}\fi

\pagestyle{empty}

\usepackage[margin=2cm]{geometry}
\setlength{\parindent}{0pt}
\setlength{\parskip}{1.5em}

\def\firstname{Antonio Guilherme }
\def\familyname{Ferreira Viggiano}
\def\subj{Personal Statement}
\def\FileAuthor{\firstname\familyname}
\def\FileTitle{\firstname\familyname~\textemdash~\subj}
\def\FileSubject{\subj}
\def\FileKeyWords{\firstname\familyname, \subj}

\renewcommand{\ttdefault}{pcr}

\usepackage{url}
\urlstyle{tt}
\ifpdf
  \usepackage[pdftex,pdfborder=0,breaklinks,baseurl=http://,pdfpagemode=None,pdfstartview=XYZ,pdfstartpage=1,colorlinks=false]{hyperref} %% added hidelinks to hyperref byAGFV
  \hypersetup{
    pdfauthor = \FileAuthor,%
    pdftitle = \FileTitle,%
    pdfsubject = \FileSubject,%
    pdfkeywords = \FileKeyWords,%
    pdfcreator = \LaTeX,%
    pdfproducer = \LaTeX}
\else
  \usepackage[dvips]{hyperref}
\fi
\usepackage{helvet}
\usepackage{ragged2e}

\begin{document}
\sffamily % for use with a résumé using sans serif fonts;

{\bfseries \FileTitle}

% universities on the planet
% leave a mark on humanity
% applying for
% which made me familiar -- NO -- semented my knowledge
% which have opened -- which opened
% I have read -- I read
% I have been reading - I read, which has helped me

Being an engineer by training and a programmer by profession, I now aspire to be a mathematician and physicist, seeking to change society through the development of science. My desire to make a difference goes back to when I applied for Mechatronics Engineering at the University of São Paulo, in Brazil, hoping to become an inventor and impact many people's lives. At the same time, during my double degree in General Engineering at the École Centrale Marseille, in France, I was exposed to many abstract lectures containing more rigorous mathematical studies of physical systems. There I started to develop a deeper interest in theoretical subjects over practical ones, which would follow me until this day. By reassessing my life objectives and personal interests, I have decided to dedicate myself to leaving a mark on the world as a discoverer, which is the main reason why I am applying for a Master of Science in Mathematical and Theoretical Physics at the University of Oxford.

% pleasure of finding things out
% create useful knowledge 
After returning to Brazil, I started working in a technology startup, imagining this would be the best way for me to leave a legacy to future generations. Because of my outstanding performance, I was invited by the chief executive officer to co-found a big data company in the development of an analytics database. Beyond the implementation of traditional algorithms and data structures, we also reviewed many open source projects and research papers, which culminated in the development of an innovative architecture, as well as in our own data storage format and messaging protocol, among other original solutions. The great adoption of the product rapidly incurred in a large number of customer demands, at a time when I spent over two months working more than twelve hours a day non-stop, including weekends and holidays. This dedication was essential for the success of the business and it demonstrates that I can face any amount of stress or work to fulfill my objectives.

Throughout my career in software development, I never ceased to deepen my knowledge of academic disciplines. It all started with Richard Feynman's popular science writing ``QED: The Strange Theory of Light and Matter'', which presented Feynman diagrams and the field of quantum electrodynamics in simple terms. Later, looking forward to extending my basis of Quantum Mechanics from the French curriculum, I began self-studying Leon van Dommelen's ``Quantum Mechanics for Engineers'' textbook in my free time, learning advanced topics such as the examination of the wave function for multiple-particle systems. More recently, wanting to learn in more detail the theory behind quantum computers, I have read through the ``Quantum Computation and Quantum Information Theory'' lecture notes from the Carnegie Mellon University. The description of quantum circuits using their matrix representations on Hilbert spaces, constantly confronted with their classical counterparts, is exhilarating to those interested in the intersection of computer science, physics and mathematics. This constant search for the ``the pleasure of finding things out'' led me to question if my occupation was completely aligned with my life goals.

Despite my accomplishments as an entrepreneur, having helped to establish a million-dollar company with more than a dozen employees, I decided to leave my position as technical lead of the organization in order to pursue a career in science. Ultimately, I have realized that even though it is possible to change the world in the private sector, the impact of one's actions are generally not as permanent as when they are applied to our comprehension of the universe. Even the most successful companies in history vanish when compared to the immortality of the discoveries of Isaac Newton or Albert Einstein, who have not only provided us with a greater understanding of physics but who have also developed whole new fields of mathematics on the way to their findings. My desire to be a part of this select group is what motivates me to join this postgraduate degree.

Oxford's Mathematical and Theoretical Physics postgraduate programme is therefore an ideal choice to expand my long-standing interest in natural philosophy and to prepare me for research in the field. Because of my previous background in quantum physics, I am particularly excited about the courses on Quantum Field Theory. Together with my familiarity with Complex Analysis' path integrals and other mathematical tools, I believe I will be able to succeed in these subjects. I am also very curious about General Relativity, as I think this will be an exciting follow up on Special Relativity and Continuum Mechanics that I have had in my exchange program, and I expect my knowledge of Tensor Calculus and Electromagnetism to be fundamental to this subject. Since I wish to get a broader understanding of nature, I intend to choose the Generalist Theoretical Physicist pathway of the curriculum. After completing the course, I would like to apply for a Ph.D. programme at the University of Oxford or another top-tier institute and continue to follow my ambition of influencing future generations as a researcher in mathematics and physics.

\end{document}
